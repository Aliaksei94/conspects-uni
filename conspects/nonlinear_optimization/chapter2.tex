\section{Nonlinear systems}

In this section we consider function $f:R\times D\rightarrow R^n$,
where $D \subseteq R^n$, and $D$ is open.

\begin{equation}\label{c2:eq}
 \dot x(t)=f(t,x(t)), \ \ t \ge t_0 \ge 0,\ \ x(t_0)=x_0
\end{equation} 

The origin $x^* \in D$ is an equilibrium point for (\ref{c2:eq}),
if $f(t,0)=0, \ \ \forall t \ge 0$.

Remark: EP (equilibrium point) $x^*=0$ can be translation of a nonzero
solution.

Suppose $\overline{y}$ is a solution of $\dot y = g(t, y)$.

Change of coordinates: $x(t)=y(t)-\overline{y}(t)$, then
$\dot x(t) = \dot y(t)-\dot{\overline{y}}(t)=
g(t, x(t)+\overline{y}(t)) - \dot{\overline{y}}(t):=f(t,x(t))$.
Since $\dot{\overline{y}}(t)=g(t,\overline{y}(t))$, then
$f(t,0)=0,\ \forall t \ge 0$.

Existence and uniqueness of solution to (\ref{c2:eq}):
\begin{itemize}
 \item if $f$ continuous, then exist local solution
 \item if $f$ continuous and locally Lipschitz in $x^*$, then
 	exist local unique solution 
\end{itemize}


Now we need new stability definitions.

\begin{Definition}
 Point $x^*=0$ is stable if $\forall \epsilon > 0$ and 
 $\forall t_0 \ge 0,\ \exists \delta>0$ s.t. from $||x_0||<\delta$
 follows $||x(t)|| < \epsilon$, $\forall t  \ge t_0$.
\end{Definition}

\begin{Definition}
 Point $x^*=0$ is uniformly stable if $\forall \epsilon > 0\ $  
 $\exists \delta>0$, s.t $\forall t_0 \ge 0,$ from $||x_0||<\delta$
 follows $||x(t)|| < \epsilon$, $\forall t  \ge t_0$.
\end{Definition}

\begin{Definition}
 Point $x^*=0$ asymptotically stable if it is stable and $\forall t_0 \ge 0\ $  
 $\exists c>0$, s.t from $||x_0||<c$
 follows $\lim_{t\to \infty} ||x(t)|| \to 0$.
\end{Definition}

\begin{Definition}
 Point $x^*=0$ uniformly asymptotically stable if it is uniformly stable and 
 $\exists c>0$, s.t $\forall t_0 \ge 0\ $ from $||x_0||<c$
 follows $\lim_{t\to \infty} ||x(t)|| \to 0$.
\end{Definition}


\begin{Definition}
 Convergence: $\forall \eta > 0 \ \ \forall t_0 \ge 0$, $\exists T>0$ such
 that $\forall t \ge t_0+T$ follows $||x(t)||<\eta$.
\end{Definition}

\begin{Definition}
 Uniform convergence: $\forall \eta > 0 \ \ \exists T>0$ such
 that $\forall t_0 \ge 0$ and $\forall t \ge t_0+T$ follows $||x(t)||<\eta$.
\end{Definition}


\begin{Example}
Consider next equation:
 \begin{equation*}
   \dot x(t) = -\frac{x(t)}{1+t}, \ \ t_0\ge 0
 \end{equation*}

 Solution $x(t)=x(t_0)\frac{1+t_0}{1+t}$. It is uniformly stable,
 because we can choose $\delta = \epsilon$. But does $x(t)$ convergence
 uniformly? Answer is no. 
\end{Example}

\begin{Definition}
 Point $x^*=0$ is globally uniformly asymptotically stable if it is uniformly stable
 with $\delta \to \infty$ for $\epsilon \to \infty$ and $\forall c,\eta\ \ $ 
 $\exists T>0$ such that $\forall t_0\ge0$ from $||x_0||<c$ follows 
 $\|x(t)\|<\eta$, $\forall t \ge t_0 + T$.
\end{Definition}


\subsection{Lyapunov's direct method}

Consider some function $V:[0,\infty)\times D\to R$, $(t,x)\to V(t,x)$ such that
$\dot V(t,x)=\frac{\partial}{\partial t}V(t,x)+\frac{\partial}{\partial x}V(t,x) f(t,x)$.

\begin{Theorem}[Lyapunov's direct method]
 Let $f:[0,\infty)\times D\to R^n$ is continuous and let $x^*=0$ be equilibrium point.
 If there is a differentiable function $V:[0,\infty)\times D\to R$ with:
 \begin{itemize}
  \item $W_1(x) \le V(t,x) \le W_2(x)$, $\forall t\ge0,\ x\in D$
  \item $\dot V(t,x)\le 0$, $\forall t \ge0,\ x\in D$
 \end{itemize}
 where $W_1,W_2:D \to R$ continuous and positive definite, then $x^*=0$ is uniformly stable.

 If further $\dot V(t,x) \le -W_3(x)$, $\forall t\ge0,\ x\in D$ with $W_3:D \to R$
 continuous and positive definite, the $x^*=0$ is uniformly asymptotically stable.

 If $D=R^n$ and $W_1$ is radialy unbounded then $X^*=0$ is globally uniformly
 asymptotically stable.
\end{Theorem}

\begin{Example}
 Consider the equation $\dot x(t) = -(1+t)x(t)$.

 Check function $V(t,x)=\frac{1}{2}x^2$ as candidate for Lyapunov's function. Then
 $W_1(x)=W_2(x)=\frac{1}{2}x^2$ and $\dot V(t,x)=-(1+t)x^2 \le -x^2(t)=: W_3(x)$.
 Then from theorem we have, that $X^*=0$  is globally uniformly asymptotically stable.
\end{Example}


\subsection{Exponential stability}

\begin{Definition}
 Point $X^*=0$ is an exponentially stable EP of (\ref{c2:eq}) if $\exists \lambda,c,k>0$
 s.t. $t \ge t_0 \ge 0$ and all $\|x_0\|<c$ follows
 $\|x(t)\|\le K\|x(t_0)\|e^{\lambda(t-t_0)}$.
\end{Definition}

Remark: from exponential stability follows uniformly asymptotically stability.

\begin{Lemma}[Auxiliary result]
 Let $\dot x(t)=f(t,x(t))$, f scalar and $\dot \xi(t)\le f(t, \xi(t))$ with 
 $\xi (t_0)\le x(t_0)$. Then $\xi (t) \le x(t)\ \ \forall t\ge t_0$.
\end{Lemma}

\begin{Theorem}
 Let $f:[0, \infty)\times D \to R^n$ be continuous and $x^*=0 \in D$ be an EP.

 If there is a differentiable function $V:[0,\infty)\times D \to R$ and constants
 $k_1,k_2,k_3,a > 0$ s.t.
 \begin{enumerate}
  \item $k_1||x||^a \le V(t,x) \le k_2\|x\|^a$, $\forall t\ge 0, \ x\in D$
  \item $\dot V(t,x) \le -k_3\|x\|^a$
 \end{enumerate}

 then $x^*=0$ is exponentially stable.

 If $D=R^n$, then $X^*$ is globally exponential stable.

 \begin{proof}
 For $c>0$ small enough, trajectories initialized in $\{x:k_2||x||^a<c\}$
 remain bounded and in D. From 1) and 2) we can conclude 
 $\dot V \le -\frac{k_3}{k_2}V$. Then from previous Lemma 
 $V(t,x(t)) \le V(t_0,x(t_0))e^{-\frac{k_3}{k_2}(t-t_0)}$. Then
 $||x(t)|| \le [from (1)] \le \left( \frac{V(t,x(t))}{k_1}\right)^{\frac{1}{a}} \le$
 $\left(\frac{V(t_0,x(t_0))e^{-\frac{k_3}{k_2}(t-t_0)}}{k_1}\right)^{\frac{1}{a}} \le$ 
 $\left(\frac{k_2||x(t_0)||^a e^{-\frac{k_3}{k_2}(t-t_0)}}{k_1}\right)^{\frac{1}{a}} =$
 $\left( \frac{k_2}{k_1}\right)^{\frac{1}{a}} ||x(t_0)||e^{-\frac{k_3}{k_2a}(t-t_0)}$
 \end{proof}
\end{Theorem}

\begin{Example}
 Consider the equation $\dot x(t) = -(1+t)x(t)$.

 Here $V(t,x)=\frac{1}{2}x^2$ then $X^*$  is exponentially stable.
\end{Example}


\subsection{Comparsion function}

\begin{Definition}
 A function $\alpha:[0,\delta)\to [0, \infty)$ is (of) "klass $K$" if it is continous,
 strictly increasing, and $\alpha(0)=0$.
\end{Definition}

\begin{Definition}
 A function $\alpha:[0,\delta)\to [0, \infty)$ is "class $K_\infty$ if $\alpha in K$
 and $\lim_{r\to\infty} \to\infty$.
\end{Definition}


\begin{Example}
 Function $\alpha(r)=\tan^{-1}(r)$ -- class $K$

 Function $\alpha(r)=r^k$ -- class $K_\infty$
\end{Example}

\begin{Definition}
 A function $\beta:[0,\delta)\times[0,\delta)\to [0, \infty)$ is "class $KL$ if it is continuous
 , $\beta(\cdot, s) \in K$ for all fixed $s$, and for each fixed $r$, $\beta(r, \cdot)$ is
 strictly decreasing: $\lim_{s\to\infty}\beta(r,s)=0$
\end{Definition}

\begin{Example}
 Function $\beta(x,s)=max(r,r^2)e^{s}$ belong class $KL$.
\end{Example}

Properties of compasion functions:

\begin{itemize}
 \item If $\alpha \in K$ on $[0, \delta)$, then $\alpha^{-1}$ is defined on 
 $[0, \alpha(\delta))$ and $\alpha^-1 \in K$.
 \item If $\alpha \in K_\infty$, then $\alpha^{-1}\in K_\infty$
 \item If $\alpha_1,\alpha_2\in K$, then $\alpha_1 \circ \alpha_2\in K$
  (same for $K_\infty$
 \item If $\alpha_1,\alpha_2\in K$, $\beta\in KL $ then 
  $\alpha_1(\beta(\alpha_2(r),s)) \in KL$
\end{itemize}

Now we conseider comparsion funcctions and stability definitions.

\begin{Lemma}
 The equilibrium $x^*=0$ of $\dot x(t)=f(t,x(t))$ is uniformly stable iff
 $\exists \alpha\in K$ and $c>0$ s.t.  $\forall t \ge t_0$,
 $\forall ||x(t_0)||<c$ and $||x(t)|| \le \alpha(||x(t_0)||).$ 

 
 \begin{proof}[(only sufficiency)]
  Given $\epsilon>0$ choose 
 $\delta<\min(c, \alpha^{-1}(\epsilon))$. Then from $||x(t_0)||<\delta$
 follows $||x(t)||\le\alpha(||x(t_0)||)<\alpha(\alpha^{-1}(\epsilon))=\epsilon$.
 \end{proof}
\end{Lemma}


\begin{Lemma}
 The equilibrium $x^*=0$ of $\dot x(t)=f(t,x(t))$ is uniformly asymptotically stable iff
 $\exists \beta\in KL$ and $c>0$ s.t. $\forall t \ge t_0$,
 $\forall ||x(t_0)||<c$ and $||x(t)|| \le \beta(||x(t_0)||, t-t_0).$ 

 \begin{proof}[(only sufficiency)] Let $\|x(t_0)\|<c$. Then 
 $\|x(t)\| \le \beta(\|x(t_0)\|, t-t_0) < \beta(c,t-t_0)$. This mean uniform convergence.
 $\|x(t)\| \le \beta(\|x(t_0)\|, t-t_0)<\beta(\|x_{t_0}\|, 0)$. This gives us uniform
 stability.
 \end{proof}
\end{Lemma}


\begin{Lemma}
 The equilibrium $x^*=0$ of $\dot x(t)=f(t,x(t))$ is uniformly asymptotically stable iff
 $\exists \beta\in KL$ and $c>0$ s.t. $\forall t \ge t_0$,
 $\forall ||x(t_0)||<c$ and $||x(t)|| \le \beta(||x(t_0)||, t-t_0).$ 

 \begin{proof}[(only sufficiency)] Let $||x(t_0)||<c$. Then 
 $||x(t)|| \le \beta(||x(t_0)||, t-t_0) < \beta(c,t-t_0)$. This mean uniform convergence.
 $||x(t)||\le\beta(||x(t_0)||, t-t_0)<\beta(||x_{t_0}||, 0)$. This gives us uniform
 stability.
 \end{proof}
\end{Lemma}

\begin{Lemma}
 The equilibrium $x^*=0$ of $\dot x(t)=f(t,x(t))$ is globally uniformly asymptotically 
 stable iff previous lemma holds for all $x_0\in R^n$.
\end{Lemma}

Now consider comparsion functios and Lyapunov functions

If $W:R^n \to R$ is continuous and positive definite, then $\forall r>0\ $
$\exists \alpha_1, \alpha_2 \in K$ s.t. $\alpha_1(||x||) \le W(x)\le \alpha_2(|x||)$ for 
all $x\in B_r(0)=\{x|||x||\le r\}$.

If $W$ is radially unbounded, then $\exists \alpha_1,\alpha_2\in K_\infty$ s.t.
$\alpha_1(||x||) \le W(x)\le\alpha_2(|x||)$ for all $x\in R^n$.

\begin{Lemma}[Auxility]
 Consider $\dot y=\alpha(y),\ \ y(t_0)=y_0>0, \ \alpha\in K$. Then 
 $\exists \beta\in KL$ s.t. $y(t)=\beta(y_0, t-t_0)$.
\end{Lemma}

Sketch of proof of Lyapunov's direct method:
\begin{equation*}
\begin{cases}
 W_1(x)\le V(t,x) \le W_2(x) \\
 \dot V \le -W_3(x)
\end{cases}
\end{equation*}


Where $W_1, W_2, W_3$ -- continuous and positive defined.

Then $\exists \alpha_1,\alpha_2,\alpha_3\in K$ such that 
$\alpha_1(||x||)\le V(t,x)\le \alpha_2(||x||)$ and $\dot V(t,x)\le -\alpha_3(||x||)$.

Proof uniform stability:

$||x(t)||\le\alpha^{-1}_1(V(t,x(t)) \le [\alpha_1 in K] \le \alpha^{-1}_1(V(t_0,x(t_0)) \le$
$\alpha^{-1}_1(\alpha_2(||x(t_0)||))$.

Proof uniform convergence

$\dot V \le -\alpha_3(||x||)\le -\alpha_3(\alpha_2^{-1}(V))$. We know, that 
$\alpha_3 \circ \alpha_2^{-1} \in K$. By comparsion lemma, $V(t,x(t))\le W(t)$, where $W$
solves $\dot W=-\alpha_3(\alpha_2^{-1}(W))$ with $W(t_0)=V(t_0,x(t_0))$. By auxility lemma
$\exists \beta\in KL$ s.t. $V(t,x(t))\le\beta(V(t_0,x(t_0)), t-t_0)$, then
$||x(t)||\le\alpha^{-1}_1(V(t,x(t)))\le\alpha^{-1}_1(\beta(V(t_0,x(t_0)), t-t_0))\le$
$\alpha_1^{-1}(\beta(\alpha_2(||x(t_0)||),t-t_0))=:$
$\bar{\beta}(||x(t_0)||,t-t_0)$. From this follows uniform asymptotic stability since
$\bar{\beta}\in KL$.


\subsection{Converse theorems}

\begin{Theorem}
 Let $X^*=0$ be an EP of $\dot x(t)=f(t,x(t))$ with $f:[0,\infty)\times R^n \to R^n$
 continously differentiable and $\frac{\partial f}{\partial x}$ bounded in $R^n$,
 uniformly in t ($||\frac{\partial f}{\partial x}(t,x)||\le L$ for all 
 $x\in R^n,\ t\ge 0,\ L>0$.

 If $x^*=0$ is globally exponentially stale, then exists differentiable 
 $V:[0,\infty)\times R^n\to R$ and $c_1,c_2,c_3,c_4>0$ s.t.
 $c_1||x||^2\le V(t,x)\le c_2||x||^2$, $\dot V(t,x)\le -c_3||x||^2$ and
 $\left\Vert \frac{\partial V}{\partial x}\right\Vert \le c_4||x||$.

\begin{proof}
Let $\Phi(\tau;t,x)$ -- solution to $\dot x(t)=f(t, x(t))$ which static
at $(t,x)$.

$V(t,x)=\int_t^{t+\delta}\Phi^T(\tau;t,x)\Phi(\tau;t,x) d\tau, \ \ \delta>0$.
Upper bound: $V(t,x)=\int_t^{t+\delta}||\Phi(\tau;t,x)||_2^2 d\tau\le$
$[exponential\ stability]\le$
$\int_t^{t+\delta}k^2 e^{-2\lambda(\tau-t)}d\tau ||x||_2^2 =$
$\frac{k^2}{2\lambda}(1-e^{-2\lambda\delta})||x||_2^2$.

Lower bound: since $\left\Vert \frac{\partial V}{\partial x}\right\Vert \le L$, then
$||f(t,x)||_2\le L||x||_2$. Thus by comparation lemma 
$||\Phi(\tau;t,x)||_2^2\ge ||x||_2^2 e^{-2L(\tau-t)}$. Set it in 
$V(t,x)\ge \int_t^{t+\delta} e^{-2L(\tau-t)} d\tau ||x||_2^2=$
$\frac{1}{2L}(1-e^{-2L\delta})||x||_2^2$.

Decrease conditions: $\dot V(t,x)=\dots\le-(1-k^2 e^{-2\lambda\delta})||x||_2^2$.
\end{proof}
\end{Theorem}