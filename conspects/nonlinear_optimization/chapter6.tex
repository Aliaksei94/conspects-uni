Remark
\begin{itemize}
 \item If (3) is sabisfied with $v_i=0$: "output - feedback passive" $\Rightarrow$ $p_i>0$
  -  "excess" of passivity, $p_i<0$ - "shortage" of passivity ($|p_i|$).
 \item If (3) satisfied with $p_i=0$: "input - feadforward passive" $\Rightarrow$ $v_i>0$
  -  "excess" of passivity, $v_i<0$ - "shortage" of passivity ($|v_i|$).
 \item Comment on terminology "output feedback passive" 
  $\dot s \le u^Ty-\rho y^Ty =$ [output feedback $u=\bar u -ky$] 
  $=\bar u^T y - (\rho+k)y^T y \le \bar u^Ty$ In other words, system can be made passive
  by output feedback.
 \item Similar for feedforward passivity, system can be made passive by feedback forward
  the input: $\bar y = y + ku$.
\end{itemize}

Remark:

Feedforward interconnection b) can be extended to allow $h_1$ or $h_2$ to explicitly depend
on $u$ $\Rightarrow$ includes static systems (controllers) e.g. state output feedback $y=h(u)$
e.g. $y_2=ku_2$ $\Rightarrow$ input-strictly passive controller.

Extension of output-feedback/input-feedforward passivity:

$$s(u,y)=u^T-\rho(y)^Ty$$

with $\rho(y) = [\rho_1(y_1),\dots,\rho_n(y_n)]^T$ with $\rho_i$ sectionnonliniarities, 
$\rho_i: \mathbb{R}\rightarrow\mathbb{R}$

\begin{Example}
\begin{equation*}
H_1: 
\begin{split}
 &\dot x_1 = x2 \\
 &\dot x_2 = -x_1^3 + x_2 + u \\
 &y = x_2
\end{split}
\end{equation*}

Take $S_1(x) = \frac{1}{4} x_1^4 + \frac{1}{2}x_2^2$ $\Rightarrow$ 
$\dot S_1 = x_1^3x_2 - x_1^3 + x_2^2 + x_2u$, define $y^2:=x_2^2$ then
$yu=x_2u$ $\Rightarrow$ output - feedback passive with shortage of passivity 1
$\Rightarrow$ $\rho_1=-1$ and $v_1 = 0$.


\begin{tikzpicture}[auto,>=latex']
    \tikzstyle{block} = [draw, shape=rectangle, minimum height=3em, minimum width=3em, node distance=2cm, line width=1pt]
    \tikzstyle{sum} = [draw, shape=circle, node distance=1.5cm, line width=1pt, minimum width=1.25em]
    \tikzstyle{branch}=[fill,shape=circle,minimum size=4pt,inner sep=0pt]
    \tikzstyle{tmp} = [coordinate]
    %Creating Blocks and Connection Nodes
    \node at (-2.5,0) (input) {};
    \node [sum] (sum) {};
    %\node at (sum) (plus) {};
    \node [block, right of=sum] (u1) {$u_1$};
    \node [block, below of=u1] (k) {$k$};
    \node [tmp, right of=u1] (tmp1) {};
    \node [tmp, right of=k] (tmp2) {};
    \node [tmp, left of=k] (tmp3) {};
    %Conecting Blocks
    \begin{scope}[line width=1pt]
         \draw[->] (input) -- (sum);
         \draw[->] (sum) -- node{$u$} (u1);
         \draw[-] (u1) -- node{$y$} (tmp1);
         \draw[-] (tmp1) -- (tmp2);
         \draw[->] (tmp2) -- node{$u_2$} (k);
         \draw[-] (k) -- node{$y_2$} (tmp3);
         \draw[->] (tmp3) -| node[pos=0.99]{$-$} (sum);
    \end{scope}
\end{tikzpicture}

$y_2u_2 = ku_2 = \gamma ku_2^2 + \frac{1-\gamma}{k}y_2^2$,  $0<\gamma<1$ $\Rightarrow$ 
$\rho 2 = \frac{1-\gamma}{k}$, $v_2 = \gamma k$

$\Rightarrow v_2+\rho 1 > 0$ for $k>1$ and $\gamma$ close enough to 1 $v_1+\rho_2 = \rho_2 > 0$.
$\Rightarrow$ with ZSO, the origin is a stable.

\end{Example}

\section{Input/Output Methods}

References: Desoev, Vidyasagar "Feedback Systes Input-output properties"

\begin{tikzpicture}[auto,>=latex']
    \tikzstyle{block} = [draw, shape=rectangle, minimum height=3em, minimum width=3em, node distance=2cm, line width=1pt]
    \tikzstyle{sum} = [draw, shape=circle, node distance=1.5cm, line width=1pt, minimum width=1.25em]
    \tikzstyle{branch}=[fill,shape=circle,minimum size=4pt,inner sep=0pt]
    \tikzstyle{tmp} = [coordinate]
    %Creating Blocks and Connection Nodes
    \node at (-1.5,0) (input) {};
    \node [block] (sys) {System};
    \node at (1.5,0) (output) {};
    %Conecting Blocks
    \begin{scope}[line width=1pt]
         \draw[->] (input) -- node{$u$} (sys);
         \draw[->] (sys) -- node{$y$} (output);
    \end{scope}
\end{tikzpicture}

\begin{equation*}
\begin{split}
&u: u\rightarrow y \\
&u,y:[0,\infty]\rightarrow \mathbb{R}^m \\
&t \rightarrow u(t), y(t)
\end{split}
\end{equation*}

\subsection{Sygnals and Systems}

\begin{itemize}
\item How to define "stability" in input/output setting?
\item Which signals are "good"?
\end{itemize}

\begin{Definition}
 $Lp$-spaces, $p\in[1,\infty]$. 
 $Lp[0,\infty) = \{\Phi:[0,\infty)\rightarrow\mathbb{R}^m, masurable|
  \int_0^\infty ||\Phi(t)||^p dt < \infty\}$
\end{Definition}

Interpretation: "finite energy sygnal" (p=2).

Remark: "measurable" = pointwise limit of a sequence of piecewise constant functions
(except on a set of measure 0)

\begin{Example}:

\begin{itemize}
 \item[-] continuous function
 \item[-] functions with "few enough" discontinuities
\end{itemize}
\end{Example}

$Lp$ is a real vector space ("signals $\Phi(\cdot)$ are vectors") i.e., for 
$\Phi, \Phi_1, \Phi_2 \in Lp$, $\alpha \in \mathbb{R}$ vector addition:
$\Phi_1+\Phi_2:t \rightarrow \Phi_1(t)+\Phi_2(t) \in Lp$. Scalar multiplication:
$\alpha \Phi:t\rightarrow \alpha\Phi(t) \in Lp$

Zero element is signal $\Phi \equiv 0$.

$Lp$ is a normed vector space with norm $||\Phi||_{Lp}=\sqrt[p]{\int_0^\infty ||\Phi(t)||^p dt}$
for $\Phi \in Lp$ $\Rightarrow$
\begin{itemize}
 \item $||\Phi||_{Lp}=0 \iff \Phi=0$, else $||\Phi||_{Lp}>0$
 \item for $\alpha \in mathbb{R}$, $||\alpha\Phi||_{Lp}=|\alpha| ||\Phi||_{Lp}$
 \item for $\Phi_1, \Phi_2 \in Lp\ $ $||\Phi_1+\Phi_2||_{Lp} \le ||\Phi_1||_{Lp}+||\Phi_2||_{Lp}$
\end{itemize}
