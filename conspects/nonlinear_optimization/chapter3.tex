\section{System with inputs}

Consider equation:

\begin{equation}\label{c3:eq}
 \dot x(t)=f(x(t),u(t)), \ \ x(0)=x_0
\end{equation}

where $f:R^n\to R^n$.

Assumption: $f$ in locally Lipschitz.

Exogenous signal $u:R\to R^n$.

Input can be "bad" (disturbance) or "good" (control).

\subsection{Input-to-state stability}

Motivation: LTI system $\dot x=Ax+Bu, \ \ x(0)=x_0$.

Solution: $x(t)=e^{At}x_0+\int_0^t e^{A(t-\tau)}Bu(\tau) d\tau$.
If A is Hurwitz, then $||e^{At}||\le c e^{-\lambda t}$ for some $c,\lambda>0$.

How large can x grow for some bounded $u$? 
$||x(t)||\le$
$||e^{At}|| ||x_0||+\int_0^t||e^{A(t-\tau)}|| ||B|| ||u(\tau)|| d\tau\le$
$e^{-\lambda t}c||x_0||+\int_0^t e^{-\lambda(t-\tau)}c||B|| ||u(\tau)|| d\tau =$
$c e^{-\lambda t}||x_0||+(1-e^{-\lambda t})$
$\frac{c}{\lambda}||B||\sup_{\tau\in[0,t]} ||u(\tau)||$.

\begin{itemize}
 \item $c e^{-\lambda t}||x_0||$ class $KL$ in $(||x_0||,t)$
 \item $(1-e^{-\lambda t})$ less than $1$
 \item $\frac{c}{\lambda}||B||\sup ||u(\tau)||$ class $K$ 
\end{itemize}

If $\sup_{\tau\in[0,t]} ||u(\tau)||$ is bounded than $\dot x$ remains bounded.
Even more: the smaller $\sup_{\tau\in[0,t]} ||u(\tau)||$, the smaller $||x(t)||$.

\begin{Definition}
 System (\ref{c3:eq}) is input-to-state stable (ISS) if $\exists \beta\in KL$,
 $\gamma\in K$ s.t. $\forall x_0\in R^n$, $\forall t \ge 0$ follows 
 $||x(t)||\le\beta(||x_0||, t)+\gamma(\sup_{\tau\in[0,t]} ||u(\tau)||)$.
\end{Definition}

Remarks:

\begin{itemize}
 \item From ISS follows O-GAS (global assymptotical stability of $x=0$ for
   $\dot x=f(x,0)$)
 \item $\gamma$ can be interpreted as "gain" w.r.t. $u$
 \item if $\lim_{t\to\infty} u(t)=0$ then $\lim_{t \to\infty}x(t)=0$
\end{itemize}

\begin{Example}
 Consider equation $\dot x = -x+xu$. System is O-GASS, not ISS (for example
 $u\equiv\alpha \Rightarrow \dot x = x (\alpha-1)$ all solution diverge).
\end{Example}

\begin{Example}
 Consider equation $\dot x =-3x+(1+2x^2)u$. System is O-GASS, not ISS (for example 
 $u\equiv1$, $x_0=2$, $x(t)=\frac{3-e^{t}}{3-2e^{t}}$ has a finite escape time.
\end{Example}

\begin{Theorem}
 Suppose that there exists a continuously differentiable function 
 $V:R^n\to R$ and $\alpha_1,\alpha_2\in K_\infty$ and $\alpha_3, \rho \in K$
 such that $\alpha_1(||x||)\le V(x)\le\alpha_2(||x||)$, $\forall x\in R^n$ and
 $\frac{\partial V}{\partial x}f(x,u)\le-\alpha_3(||x||)$,
 $\forall x: ||x||\ge\rho(||u||)$. Then (\ref{c3:eq}) is ISS with 
 $\gamma = \alpha_1^{-1} \circ \alpha_2\circ\rho$

 \begin{proof}
 Idea: same as Lyapunovs direct method when $x$ is "outside" of ball
 $\{x | ||x||\le\rho(||u||)\}$

 TODO Picture
 \end{proof}
\end{Theorem}


\begin{Example}
 Consider equality $\dot x = -x^3+u$. Let $V(x)=\frac{1}{2}x^2$, then
 $\dot V = -x^4+xu=[0<\Theta<1]=-(1-\Theta)x^4-\Theta x^4+xu\le$
 $-(1-\Theta)x^4$ for all $x: ||x||\ge\left(\frac{||u||}{\Theta}\right)^\frac{1}{3}$.
 Thus, system is ISS with 
 $\gamma(v)=\rho(v)=\left(\frac{v}{\Theta}\right)^\frac{1}{3}$.
\end{Example}


Remarks:
\begin{itemize}
 \item Existence of $V$ is both neccessary and sufficient for ISS;
 \item Takes place $\frac{\partial V}{\partial x}f(x,u) \le -\alpha_4(||x||)+\alpha_5(||u||), \forall x,u$ for some $\alpha_4,\alpha_5\in K$;
 \item % TODO Picture
  If $x_1=0$ is a globally asymptotically stable EP of $\Sigma_1$ and $\Sigma_2$ is ISS w.r.t. "input" $x_1$, then $(x_1,x_2)=(0,0)$ is a globally asymptotically stable EP for the cascaded system.
\end{itemize}

\begin{Theorem}
 Assume that:
 \begin{itemize}
     \item $f$ is globally Lipschitz;
     \item $x = 0$ is a globally exponentially stable EP for $\dot x = f(x,0)$
 \end{itemize}
 Then the system (\ref{c3:eq}) is ISS.
 
 \begin{proof}
 Sketch:
 $\exists$ continuous differentiable $V$:
 $$c_1||x||^2 \le V(x) \le c_2||x||^2$$
 $$ \frac{\partial V}{\partial x}f(x,0)) \le -c_3||x||^2$$
 $$||\frac{\partial V}{\partial x}|| \le c_4||x||$$
 Then: \\
  $\frac{\partial V}{\partial x}f(x,u)=\frac{\partial V}{\partial x}f(x,0)+\frac{\partial V}{\partial x}(f(x,u)-f(x,0) \le -c_3||x||^2+c_4||x||L||u|| = -c_3(1-\theta)||x||^2-\theta||x||^2+c_4L||x||||u|| \le -c_3(1-\theta)||x||^2$ \\
  if $||x|| \ge \frac{c_4L}{\theta c_3}||u||$.
 \end{proof}
\end{Theorem}

\subsection{Control Lyapunov functions}

Motivation: Lyapunov theory for control systems.

(input affine systems) \\
$\dot x = f(x) + \sum_{i=1}^{m}g_i(x)u_i = f(x) + G(x)u, \\ f:R^n \to R^n, g:R^n \to R^n, G:R^n \to R^{n \times m}$ \\
$u:t \to u(t), R \to R^m$ is a control signal (decision variable).

\begin{Definition}
 A function $V:R^n \to R$ is a control Lyapunov function (CLF) if it's differentiable positive definite, radially unbounded and 
 \begin{equation} \label{c3:s2:1}
     \forall x \neq 0 \ \ \inf_u (\nabla V(x) \cdot (f(x)+G(x)u)) < 0 
 \end{equation}
\end{Definition}

Remark: \\
Concept can be generalized to systems $\dot x = f(x,u)$. Then \ref{c3:s2:1} becomes
$$\forall x \neq 0 \ \ \inf_u (\nabla V(x) \cdot f(x,u)) < 0 $$

\begin{Theorem}[Artstein]
 There exists $k:R^n \to R^m$ (state feedback) which is continuous on $R^n \backslash \{ 0 \}$ s.t. $x^*=0$ is globally asymptotically stable EP for $\dot x = f(x)+G(x)k(x)$ iff there exists a CLF.
\end{Theorem}

How to find CLFs?

Proposition: \\
Condition (\ref{c3:s2:1}) is equivalent to
 \begin{equation} \label{c3:s2:2}
     \forall x \neq 0, \ \ \frac{\partial V}{\partial x}G(x) = 0 \implies L_fV(x) < 0 
 \end{equation}

Remark: 
$$\frac{\partial V}{\partial x}G(x) = (\nabla V(x)g_1(x), \dots \nabla V(x)g_m(x)) =: L_GV(x)$$
$$(\ref{c3:s2:2}) \iff \forall x \neq 0, \ \ L_fV(x) \ge 0 \implies L_GV(x) \neq 0$$
\begin{proof}
    $\Longleftarrow$: \\
    Assume (\ref{c3:s2:2}) holds. Then: \\
    $$\inf_u (\nabla V(x) \cdot (f(x)+G(x)u)) = \inf_u L_fV(x)+L_GV(x)u < 0$$ \\
    Why?
    \begin{itemize}
        \item If $L_GV(x) = 0$, then by (\ref{c3:s2:2}) $L_fV(x) < 0$;
        \item If $L_GV(x) \neq 0$, then (at least) for one $i$ we have $\nabla V(x) \cdot g_i(x) \neq 0 \implies$ set $u_i = -c \nabla V(x) \cdot g_i(x)$.
    \end{itemize}
    
    $\Longrightarrow$: \\
    If (\ref{c3:s2:1}) holds for some $x$ with $L_GV(x)=0$, then we must have $L_fV(x) < 0$.
\end{proof}

Example (discontinuous control):
$$\dot x = \left\{
                \begin{array}{ll}
                  1-u, \ \ u \ge 1\\
                  -1-u, \ \ u \le -1\\
                  0, \ \ else
                \end{array}
              \right.$$
If you want to move the system you need to apply control $|u| \ge 1$. \\
Using $$u(x) = \left\{
                \begin{array}{ll}
                  x+1, \ \ x>0\\
                  x-1, \ \ x \le 0
                \end{array}
              \right.$$
results in closed loop $\dot x = -x$ - asymptotically stable.\\
$V(x) = x^2$ is a CLF.

\begin{Theorem}
 There exists a continuous $k:R^n \to R^m$, smooth on $R^n \backslash \{ 0 \}$ s.t. $x^*=0$ is globally asymptotically stable EP for $\dot x = f(x)+G(x)k(x)$ iff:
 \begin{itemize}
     \item there exists a (smooth)CLF V;
     \item $\forall \varepsilon > 0 \ \ \exists \delta > 0: \ \ \forall x: 0<||x||<\delta$ \\
     $\exists u \in R^m: ||u|| < \varepsilon$ s.t. $L_fV(x)+L_GV(x)u < 0$
 \end{itemize}
\end{Theorem}

How to construct a globally stabilizing state feedback $k$ from knowledge of a CLF?

"Sontag's formula" \\
Fix $c \ge 0, a(x):=L_fV(x), b(x):=(L_GV(x))^T$
$$k(x) = \left\{
                \begin{array}{ll}
                  -cb(x)-\frac{a(x)+\sqrt{a(x)^2+(b(x)^Tb(x))^2}}{b(x)^Tb(x)}b(x)^T, \ \ b(x) \neq 0\\
                  0, \ \ b(x)=0
                \end{array}
              \right.$$
              
Proposition:
Let $V:R^n \to R$ be a CLF and $k$ as above. Then $x^*=0$ is globally asymptotically stable EP for $\dot x = f(x)+G(x)k(x)$
\begin{proof} 
    $\dot V = L_fV(x)+L_GV(x)k(x) = a(x) - cb(x)^Tb(x) - \frac{a(x)+\sqrt{a(x)^2+(b(x)^Tb(x))^2}}{b(x)^Tb(x)}b(x)^Tb(x) = -cb(x)^Tb(x) - \sqrt{a(x)^2+(b(x)^Tb(x))^2} < 0 \ \ \forall x \neq 0$ s.t. $L_GV(x) \neq 0$ \\
    
    $\dot V = L_fV(x)+L_GV(x) \cdot 0 < 0\ \ \forall x \neq 0$ s.t. $L_GV(x) = 0$ (since $V$ is CLF) \\
    
    $\implies V$ - Lyapunov function $\implies \dots$
\end{proof}

Remarks:
\begin{itemize}
    \item Sontag's formula is smooth on $R^n \backslash \{ 0 \}$;
    \item Sontag's formula is continuous at $x=0$ iff small control property holds.
\end{itemize}

\begin{equation*}
\forall x \neq 0: \ \inf_{u} \frac{\partial V}{\partial x} f(x,u) < 0 \ \dot{x} = f(x) + G(x)u
\end{equation*}
So this leads to
\begin{equation*}
\forall x \neq 0 L_G V(x) = 0 \Rightarrow L_fV(x) \neq 0
\end{equation*}
Remark: The last formula is "optimal" if minimize:
\begin{equation*}
\int_{0}^{\infty}\frac{1}{2}p(x)b(x)^Tb(x) + \frac{1}{2p(x)}u^Tudt
\end{equation*}
$b(x) := (L_GV(x))^T$

where $c > 0$

\begin{equation*}
p(x) = \left \{ 
\begin{tabular}{cc} 
$c + \frac{a(x) + \sqrt{a(x)^2 + (b(x)^Tb(x))^2}}{b(x)^Tb(x)}$ & $b(x) \neq 0$ \\ 
$c$ & $b(x) = 0$ 
\end{tabular} 
\right.
\end{equation*}

It still works if $u = \lambda h(x)$ with $\lambda \in [\frac{1}{2}; \infty)$ is applied (large "gain margin")