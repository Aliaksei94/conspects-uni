\section{Differential equations}

Consider differential equality
\begin{equation}\label{c1:eq}
\frac{d}{dt}x(t)=\dot x(t)=f(x(t)), \ x(0)=x_0 
\end{equation}

Where $f:D\rightarrow R^n, \ D\subset R^n$ is open, [here we should explain,
what means open set].

Solution to \ref{c1:eq} $x:I_{x_0} \rightarrow D, \ t\rightarrow x(t)$ is
differentiable

Interval existence solution

Questions:

\begin{itemize}[label=$\#$]
\item existence of solution
\item "how large" is $I_{x_0}$
\item uniquence of solution
\end{itemize}



Usaly we will add some restrictions on $f$ functions, like continuous.

\subsection{Existence of solutions}

%\begin{Definition}
Function $f:D \rightarrow R^n$ is contiuous at $x' \in D$ if for
$\forall \epsilon > 0 \ \exists \delta>0$ such that for $\forall x \in D$,
$\|x-x'\|<\delta => \|f(x)-f(x')\| < \epsilon$

Function $f:D \rightarrow R^n$ is continuous on D if it's 
continuous at $\forall x' \in D$
%\end{Definitions}

%\begin{Theorem}[Piano]
If $f:D \rightarrow R^n$ continuous, then for each $x_0 \in D$\ 
$\exists x:(-\epsilon,\epsilon) \rightarrow D, \ \epsilon > 0$
satisfying (\ref{c1:eq}).
%\end{Theorem}

Further, given a compact sed $U \subset D$, then 
$\exists \alpha > 0$ s.t. $\forall x_0 \in U 
\ \exists x:(-\epsilon,\epsilon) \rightarrow D$ satisfying (\ref{c1:eq}).


%\begin{example}
Consider equation $\dot x(t) = x(t)^2, \ \ x(0)=x_0=0$. Solution 
$x(t)=-\frac{1}{t-c}, \ \ c=\frac{1}{x_0}$. In this example
solution exist in interval $(-c, c)$.
%\end{example}


But, what about the number of solutions? Which conditions we should
add to garanty uniquence of solution?


\subsection{Uniquence of solutions}

%\begin{Definitions}
Function $f:D \rightarrow R^n$ is locally Lipshitz (continuous???) on D
if $\forall x \in D$ there is a neighborhood $N(x) \subset D$ and
$\exists L > 0$ s.t. 
\begin{equation}\label{c1:lip}
||f(x_1)-f(x_2)|| \le L ||x_1 - x_2|| 
\end{equation}

For all $x_1,x_2 \in N$.
%\end{Definitions}

\begin{itemize}
\item Lipschiz on $W \in D$ if (\ref{c1:lip}) holds $\forall x_1,x_2 \in W$ 
(with same L)
\item globally Lipschitz if (\ref{c1:lip}) holds $\forall x_1, x_2 \in R^n$
(with same L)
\end{itemize}


We have
\begin{itemize}[label=$\#$]
 \item localy Lipschitz functions are continuous
 \item differenciable functions are locally Lipschitz
 \item locally Lipschitz functions are Lipschitz on each compact subset of D
 	(Khalil Ex 3.19)
\end{itemize}

%\begin{Lemma}[Cromwall]
Suppose that $0 \le \phi(t) \le c + L \int^t_0 \phi(\tau) d \tau$, $c,L > 0$, $\phi$ -- 
continuous. Then $\phi(t) \le c \exp{Lt}$.
Proof. $c + L\int^t_0 \phi(\tau) d \tau := \psi(t)$, $\dot \psi(t) = L\phi(t) \le L \psi (t)$.

Consider $\frac{d}{dt} \left(\psi(t) \exp{-LT}\right) = \exp{-Lt}
\dot \psi(t)-L \psi(t) \left(\right) \le 0$, thus $\psi(t) \exp{-LT}$ is
decreased, and as a result we have $\phi(t)\exp{-Lt}\le \psi(t)\exp{-Lt}\le \psi(0)=c$ 
%\end{Lemma}

%\begin{Theorem}[Picard Lindelof]
If function $f:D \rightarrow R^n$ is localy Lipschitz then for $\forall x_0 \in D$
\ $\exists ! x:(-\epsilon, \epsilon) \rightarrow D, \ \epsilon > 0$
satisfying (\ref{c1:eq}).

Proof:

* existence from Piano theorem

Proof of uniqueness

Consider two solutions $x_1(.)$ and $x_2(.)$ to (\ref{c1:eq}). $\dot x_1-\dot x_2=f(x_1)-f(x_2)$,
$x_1(0)=x_2(0)$. Then we can integrate equality: 
$x_1(t)-x_2(t)=\int^t_0 f(x_1(\tau))-f(x_2(\tau)) d\tau$.  
$|x_1(t)-x_2(t)| \le \int^t_0 |f(x_1(\tau))-f(x_2(\tau))| d\tau \le 
L \int^t_0 |f(x_1(\tau))-f(x_2(\tau))| d\tau$. Now we can apply Cromwall's lemma with $c=0$
and $\phi(t)=|x_1(t)-x_2(t)|$, then $\phi(t)\le 0$, then $x_1(t)=x_2(t), \ \forall t 
\in (-\epsilon,\epsilon)$
%\end{Theorem}

%\begin{example}
\begin{equation*}
    \dot x = 
    \begin{cases}
      \sqrt{x}, & \mbox{if } x \ge 0 \\
      0, & \mbox{else } x<0 
    \end{cases}
\end{equation*}

\begin{equation*}
    Solutions\ x(t) =
    \begin{cases}
      \frac{1}{4}(t-c)^2, & \mbox{if } t \ge c \ge 0 \\
      0, &\mbox {else} 
    \end{cases}
\end{equation*}
%\end{example}

Global existance $\&$ uniqueness
\begin{itemize}
 \item sufficient condition: $f$ globally Lipschitz
 \item another sufficient condition: solution entirely lies in a coplex set
 \item forward completeness has equivalent Lyapunov-like characterization: system is
      forward-complete iff $\exists$ solution $V:R^n \rightarrow R \ge 0$ s.t.
      $\frac{\partial V}{\partial x} f(x) \le V(x)$, $\forall x \in R^n$
\end{itemize}


\subsection{Lyapunov stability}

If functions $\dot V(x) < 0$, $\forall x \in D\\ \{0\}$, then $x^*$ is
asymptotically stable.

%\begin{Definition}
Equilibrium point $x=0$ is stable if $\forall \epsilon > 0\ \exists \delta > 0$
s.t. from $\|x_0\| < \delta$ follows $\|x(t)\| \le \epsilon, \ \forall t \ge 0$.
%\end{Definition}

%\begin{Definition}
Equilibrium point $x=0$ is asymptotikaly stable if it stable and exist $\delta>0$
s.t. from $\|x_0\|<\delta$ follows 
$\lim_{t\rightarrow \infty}x(t) \rightarrow 0$.
%\end{Definition}


%\begin{Theorem}[Lyapunov's direct method]
Let $x^*=0 \in D$ be an equilibrium point of (\ref{c1:eq}), i.e., $f(0)=0.$ Let
$f: D \rightarrow R^n$ is continious. If there exist a differetiable 
$V:D\rightarrow R$ s.t. 
\begin{enumerate}
\item $V(x^*)=0$, $V(x)>0$, $\forall x \in D\\ \{0\}$
\item $\dot V(x) = \frac{\partial V}{\partial x}f(x) \le 0$, $\forall x \in D$
\end{enumerate}

then $x^*=0$ is stable.

Proof. Fix compact $U=\{x:V(x)\le c\}$ s.t. $U\in D$. By Piano: exist $\alpha > 0$
s.t. any solution $x$ with $x_0 \in U$ exists at least on the interval
$[0,\alpha)$.

TODO proof is not full

%\end{Theorem}

Lyapunovs direct method gives us:
\begin{itemize}
 \item stability
 \item convergence (if $V<0$)
 \item subset of the region of attraction (all compact $U=\{x:V(x)\le c\} \in D$)
 \item existance of solution for all times
\end{itemize}





 

