\section{Robust MPC}

Consider linear (discrete-time) sytem:
$x(t+1) = Ax(t) + Bu(t) + w(t)$
in short $x^+=Ax+Bu+w$

Constraints: $x(t) \in X, u(t) \in U, \forall t = 0,1...$

Bound on $w$:
$W$ is a compact,convex set which contains $0$. $w(t) \in W \ \forall t =0,1,...$

Main idea:
Use additional error feeedback s.t. real systems state contained in a "tube" around some nominal system state.

Repetition of QI-MPC in discrete time:

Nominal system:
\begin{equation*}
z^+ = Az+Bv
\end{equation*}

At time $t$, given $z(t)$, solve 
\begin{equation*}
\min_{v(\cdot | t)} \hat J(z(t),v(\cdot|t)) = \sum_{i=t}^{t+N-1}L(z(i|t),v(i|t)) + F(z(t+N|t))
\end{equation*} 

s.t. 
\begin{equation*}
z(i+1|t) = Az(i|t)+Bv(i|t), z(t|t) = z(t)
\end{equation*}
\begin{equation*}
z(i|t) \in Z, v(i|t) \in V, t \leq i \leq t+N-1
\end{equation*}
\begin{equation*}
z(t+N|t) \in Z^f \subseteq Z
\end{equation*}

$\Rightarrow$ optimizer $V^*(\cdot|t)$, optimal value function $\hat J^*(z(t))$

Assumption 1:
\begin{itemize}
\item Cost is quadratic $L(z,v) = z^TQz + v^tRv, Q,R > 0$
\item There exists a local auxiliary controller $k^{loc} = Kx$ s.t.
\begin{enumerate}
\item $Z^f$ is invariant with $Z^+=(A+BK)z$, $A_k = A+BK$, i.e. $A_kZ^f \subseteq Z^f$
\item $Kz \in V \forall z \in Z^f$
\item $F(A_kz)-F(z) \leq - L(z,Kz) \forall z \in Z^f$
\end{enumerate}
\end{itemize}

From Assumption 1 it follows (as in continuous time) that 

\begin{equation*}
\hat J^*(z(t+1)) - \hat J^*(z(t)) \leq - L(z(t), v_{MPC}(t))
\end{equation*}
Since $L$ is quadratic, there exists constants $c_2 > c_1 > 0$ s.t. $\forall z \in Z_N $ - feasible set
\begin{enumerate}
\item $c_1|z|^2 \leq \hat J^*(z)$
\item $\hat J^*(z^+) - \hat J^*(z) \leq - c_1 |z|^2$
\item $\hat J^*(z) \leq c_2|z|^2$
\end{enumerate}

Why is (3) true?

From Assumption 1.3 $\forall z \in Z^f$
\begin{equation*}
\hat J^*(z) \leq \hat J(z,Kz(\cdot)) = \sum_{i=1}^{N-1}L(z(i),Kz(i)) + F(z(N)) \leq 
\end{equation*}
N times apply Assumption 1.3
\begin{equation*}
\leq F(z) = z^TPz \leq \lambda_{max}(P)|z|^2
\end{equation*}

Influence of disturbance:
\begin{Definition}
Minkowski set addition: 
\begin{equation*}
A,B \subseteq \mathbb{R}^n A \oplus B = \{ a+b | a \in A, b\in B\}
\end{equation*}

Pontryagin set difference:
\begin{equation*}
A, B \subseteq \mathbb{R}^n A \ominus B = \{ a \in \mathbb{R}^n| a+b \in A, \forall b \in B\}
\end{equation*}
\begin{equation*}
(A \ominus B) \oplus B \subseteq A
\end{equation*}
\begin{equation*}
A \subseteq (A \oplus B) \ominus B
\end{equation*}
\end{Definition}
\begin{Definition}

Robust positively invariant set (RPI set):

$S$ is RPI set for $x^+=Ax+w$ if $AS \oplus W \subseteq S$ (or equivalently $Ax+w \in S \forall x \in S, \forall w \in W$)
\end{Definition}
\begin{Example}

$x^+ = 0.5x + w$. $w \in [-5,5]$. So RPI set: $S = [-20, 20]$, minimal RPI set: $S = [-10,10]$
\end{Example}

Minimal RPI set:
\begin{equation*}
S_{\infty} = \sum_{i = 0}^{\infty} A^iw
\end{equation*}
(Minkowski set addition), min. RPI set exists and is bounded if $A$ is Schour table.

Why?

Current state at time $t$ is $x$,

possible states at time $t+1$: $Ax \oplus W$
                        
$t+2$: $A(Ax \oplus W) \oplus W = A^2x \oplus AW \oplus W$
  
       ........
        
$t+j$: $A^jx \oplus \sum_{k=0}^{j-1}A^kw$

$\Rightarrow$ by choosing $j$ large enough we can reach any state in $S_{\infty}$

$\Rightarrow$ any RPI set must satisfy $S_{\infty} \subseteq S$

Remains to show: $S_{\infty}$ is an RPI set
\begin{equation*}
AS_{\infty} \oplus W = A\sum_{i=0}^{\infty}A^iw \oplus W = \sum_{i=1}^{\infty}A^iw \oplus W = S_{\infty} 
\end{equation*}

$S_{\infty}$ in  general is difficult to compute 

$\Rightarrow$ can compute invariant outer approximations of $S_{\infty}$ (with bounded complexity)

\begin{Example} 

Calculate RPI 
\begin{equation*}
S_{\infty} = \sum_{i=0}^{\infty} A^iw
\end{equation*}
For the system given and bounded disturbances
\begin{equation*}
x^+=\frac{1}{2}x + w, \ w \in [-5,5]
\end{equation*}
\begin{equation*}
S_{\infty} = \sum_{i=0}^{\infty}(\frac{1}{2})^i [-5,5] = [-10, 10]
\end{equation*}
\end{Example}

Central idea in tube-based MPC

Use additional error feedback around some nominal input:
\begin{equation*}
u_{MPC} = v_{MPC}(x) + K (x - z)
\end{equation*}

Proposition 1

Let $x^+ = Ax + Bu + w$ and $z^+ = Az + Bv$. If $ x \in Z \oplus S$ and $ u = v + K(x - z)$, then $ X^+ \in Z^+ \oplus S$ (RPI set for $x^+ = (A+BK)x + w$)

image to be inserted

\begin{proof}

Let $e(t) := x(t) - z(t) \rightarrow$
\begin{equation*}
e^+ = x^+ - z^+ = Ax + B(v + K(x-z)) + w - Az - Bv = (A+BK)e + w
\end{equation*}
As $S$ is RPI for $e^+ = A_ke + w$, we obtain $e \in S \Rightarrow e \in S \ \forall w \in W$

Hence $x \in Z \oplus S \Rightarrow x^+ \in Z^+ \oplus S \ \forall w \in W$

\end{proof}

Robust MPC scheme 

MPC problem for robust tube-based MPC: At time $t$, given $x(t)$, solve 

\begin{equation*}
\min_{z(t|t), v(\cdot|t)} J(x(t),v(\cdot|t)) = \sum_{i = 1}^{t+ N -1}L(z(i|t), v(i|t)) + F(z(t+N|t))
\end{equation*}
\begin{equation*}
s.t. z(i+1|t) = Az(i|t) + Bv(i|t)
\end{equation*}  
\begin{equation*}
z(i|t) \in Z = X \ominus S 
\end{equation*}
\begin{equation*}
v(i|t) \ \in V = U \ominus KS
\end{equation*}
\begin{equation*}
t \leq i \leq t + N -1
\end{equation*}
\begin{equation*}
z(t+N|t) \in Z_N \subseteq Z 
\end{equation*}
Initial condition $x(t) \in z(t|t) \oplus S$

$\rightarrow$ optimizer: $z^*(t|t), v^*(\cdot|t) \rightarrow$ optimal value function $J^*(x(t))$

$\rightarrow$ applied input: $u(t) = v^*(t|t) + K(x(t) - z^*(t|t))$

Important: Tightened input/state constraints for the nominal predictions ensure fulfilment of original input/state constraints for real (disturbed) closed-loop system.

Properties of robust MPC scheme (in the following $z^*(x(t)):=z^*(t|t)$)
\begin{itemize}
\item a feasible set $X_N = Z_N \oplus S \subseteq X$
\item $J^*(x) = \hat{J^*}(z^*(x))$ by definition of $J^*$ and $\hat J^*$
\item $J^*(x) = 0 \ \forall x \in S$
\end{itemize}    

Why? 

If $x \in S$, then $z(x) = 0$ and $v(\cdot|t) = 0$ is a feasible solution. Hence $J^*(x) \leq \hat{J}(0,0) = 0$

$\Rightarrow J^*(x) = 0$ and $z^*(X) = 0$

"$S$ serves an origin for the disturbed system"

\begin{Theorem}Suppose that Assumption 1 holds and the robust MPC problem is feasible at $t = 0$.

Then: 
\begin{enumerate}[label=(\roman*)]
\item robust MPC problem is recursively feasible
\item closed-loop system robustly exponentially converges to $S$
\item closed-loop system satisfies input/state constraints, i.e. $x(t) \in X$, $u(t) \in U \ \forall t = 0,1 ...$
\end{enumerate} 

\begin{proof}

i) Consider candidate solution at time $t+1$ 
\begin{equation*}
\tilde{V}(i|t+1) = \left \{
  \begin{tabular}{c}
  $v^*(i|t) \ t+1 \leq i \leq t+ N -1$ \\
  $k^{loc}(z^*(t+N|t)) \ i = t + N$
  \end{tabular}
\right .
\end{equation*}
\begin{equation*}
\tilde{z}(t+1 | t+1) = z^*(t+1|t)
\end{equation*}

it is feasible because $x(t+1) \in z^*(t+1|t) \oplus S$ by proposition 1

image to be inserted


iii) follows from Proposition 1 + definition of tightened constraints


ii) from (1-3) inequalities described below

\begin{enumerate}
\item $\hat{J^*}(z) \geq C_1|z|^2$
\item $\hat{J^*}(z^+) - \hat{J^*}(z) \leq - c_1 |z|^2$
\item $\hat{J^*}(z) \leq c_2|z|^2$
\end{enumerate}

$J^*(x) = \hat{J^*}(z^*(x))$

we obtain the following $\forall x \in X_N$
\begin{enumerate}
\setcounter{enumi}{3}
\item $J^*(x) = \hat{J^*}(z^*(x)) \geq^{(1)} c_1|z^*(x)|^2$
\item $J^*(x) = \hat{J^*}(z^*(x)) \leq^{(3)} c_2|z^*(x)|^2$
\end{enumerate}

So now we will show convergence to $0$
\begin{equation*}
J^*(x(t+1)) - J^*(x(t)) = \hat{J^*}(z^*(x(t+1))) - \hat{J^*}(z^*(x(t))) \leq
\end{equation*}
\begin{equation*}
\leq \hat{J^*}(z^*(x(t+1|t))) - \hat{J^*}(z^*(x(t))) \leq^{(2)}
\end{equation*}
\begin{equation*}
-c_1|z^*(x(t))|^2 \leq - \frac{c_1}{c_2}J^*(x(t))
\end{equation*}
$\Rightarrow$
\begin{equation*}
J^*(x(t+1)) \leq (1 - \frac{c_1}{c_2})J^*(x(t))
\end{equation*} 
where $\gamma := 1 - \frac{c_1}{c_2}, \ \gamma \in (0,1)$

\begin{equation*}
J^*(x(i)) = \gamma^iJ^*(x(0)) \leq^{(5)} c_2\gamma^i|z^*(x(0))|^2
\end{equation*}
$\Rightarrow^{(4)}$
\begin{equation*}
|z^*(x(i))| \leq \sqrt{\frac{c_2}{c_1}}\sqrt{\gamma}^i|z^*(x(0))|
\end{equation*}
$\Rightarrow z^*(x(t))$ exponentially converges to $0$.

Recall: $x(i) \in z^*(x(i)) \oplus S \Rightarrow$
\begin{equation*}
|x(i)|_S \leq |z^*(x(i))| \leq \sqrt{\frac{c_2}{c_1}}\sqrt{\gamma}^i|z^*(x(0))|
\end{equation*}
\end{proof}
\end{Theorem}

$|x(i)|_S$ - point-to-set distance

Extensions:
\begin{itemize}
\item Nonlinear systems: difficult to compute RPI sets
\begin{itemize}
\item approaches based on input-to-state stability(ISS)
\item approaches which apply MPC two times:
\begin{itemize}
\item first for nominal input
\item to determine local error feedback(Rawlings and Mayne chapter 3-6)
\end{itemize}
\end{itemize}
\item Linear systems with parametric uncertainties
\begin{equation*}
x(t+1) = A(t)x(t) + B(t)u(t)
\end{equation*}
$(A(t),B(t)) \in \rho : con{(A_j,B_j), j = 1,...,J} \ \forall \geq 0$ 
\end{itemize}
Note. $co$- convex

Define: $\bar{A} := \frac{1}{J}\sum_{i=0}^{J}A_i$, $\bar{B} := \frac{1}{J}\sum_{i=0}^{J}B_i$
\begin{equation*}
x(t+1) = \bar{A}x(t) + \bar{B}u(t) + w(t)
\end{equation*}
\begin{equation*}
w(t) \in W := {(A-\bar{A})x + (B-\bar{B})u | (A,B) \in \rho, x \in X, u \in U}
\end{equation*}
$W$ is compact if $X$,$U$ are compact

$\rightarrow$ can apply tube MPC as before
but: can slow down more!

If $\rho$ is "small enough", closed-loop asymptotically to zero

Intuition: $x$ converges to the RPI set $S \rightarrow W$ gets smaller 

$\rightarrow x$ converges to RPI set

Invariant approximations of the minimal RPI set $S_{\infty}$ is difficult to compute
\begin{equation*}
S_{\infty} := \sum_{i=0}^{\infty}A^iw
\end{equation*}

Define $S_k := \sum_{i=0}^{k-1}A^iw \ k \geq 1$

In general, $S_k$ for a finite $k$ are not RPI sets (this is the case if only if $A$ is nilpotent)

\begin{Theorem}

If $0 \in int(W)$ and $A$ is Schur, then there exists an integer $k > 0$ and $\alpha \in [0,1)$ s.t.
\begin{equation}\label{RPI_set_subset_with_alpha}
A^kW \subseteq \alpha W
\end{equation} 

If (\ref{RPI_set_subset_with_alpha}) holds, then 
\begin{equation*}
S(\alpha,k) := (1-\alpha)^{-1}S_k
\end{equation*}
is an RPI set for the system $x^+ = Ax + w$

\begin{proof}

i)(\ref{RPI_set_subset_with_alpha}) is a direct consequence of our assumptions
ii) want to show that $AS(\alpha,k) \oplus W \subseteq S(\alpha,k)$
\begin{equation*}
AS(\alpha,k) \oplus W = (1-\alpha)^{-1} \sum_{i=1}^{k} A^iW \oplus W= 
\end{equation*}
\begin{equation*}
= (1 - \alpha)^{-1} A^kW \oplus \sum_{i=1}^{k-1}A^iW(1-\alpha)^{-1} \oplus W
\end{equation*}

As far as $A^kW \subseteq \alpha W$ by (\ref{RPI_set_subset_with_alpha})

\begin{equation*}
(1-\alpha)^{-1}\alpha W \oplus W \oplus \sum_{i=1}^{k-1}A^iW(1-\alpha)^{-1}
\end{equation*}

As $(1-\alpha)^{-1}\alpha W \oplus W = [(1-\alpha)^{-1} \alpha + 1]W = (1-\alpha)^{-1}W$

Then we get 
\begin{equation*}
= (1-\alpha)^{-1}\sum_{i=0}^{k-1}A^iW = S(\alpha,k)
\end{equation*}
\end{proof}
\end{Theorem}

Remark:
\begin{itemize}
\item for a given $k$ s.t. (\ref{RPI_set_subset_with_alpha}) can be satisfied, we want to find the smallest possible $\alpha$ ("small scaling factor")
\item for a given $\alpha$, one wants to find the smallest possible $k$ s.t. (\ref{RPI_set_subset_with_alpha}) holds ("low complexity" of RPI set)

$\Rightarrow$ tradeoff between small $\alpha$ and small $k$ needs to be found 
\item one can determine "how good" $S(\alpha,k)$ is compared to $S_{\infty}$

$\Rightarrow$ can specify suboptimality degree of approximation a priori

Possible algorithm to determine RPI set
\begin{enumerate}
\item fix $\alpha \in (0,1)$ and $k > 0$ (integer)
\item check whether (\ref{RPI_set_subset_with_alpha}) holds:
\begin{itemize}
\item if yes: $S(\alpha,k)$ is a RPI set
\item if not: set $k := k+1$ and go to (2)
\end{itemize}
\end{enumerate}
\end{itemize}   