\section{Economic MPC}

Motivation: (setpoin) stabilization is often not primary control objective 
$\Rightarrow$ "general" cost function $L$.

Assumption: $L:\mathbb{R}^n\rightarrow\mathbb{R}^n$ is continuous (need not
be positive defined) $X,U$ is compact.

\begin{Definition}
 Optimal steady-state $(x_s,u_s)=\arg\min_{x^+=f(x,u);\ x\in X;\ u\in U}L(x,u)$
\end{Definition}

\begin{Example}
 \begin{equation*}
  \begin{split}
   &x^+=xu \\
   &X=U=[-\delta,\delta] \\
   &L(x,u)=g(x)+(u+1)^2
  \end{split}
 \end{equation*}

 Optimal steady state is $(x_s,u_s)=(0,-1)$

 Optimal operating condition:

 \begin{equation*}
  \begin{split}
   &x=(-1;+1;-1;\dots) \\
   &u=(-1;+1;-1;\dots)
  \end{split}
 \end{equation*}
\end{Example}


Economic MPC problem

At each time $t$, given $x(t)$, we should minimize $J(x(t),u(\cdot|t)$ s.t.

$$J(x(t),u(\cdot|t)\sum_{k=t}^{t+N-1}L(x(k|t),u(k|t))$$

With conditions:

\begin{equation*}
 \begin{split}
  &x(k+1|t) = f(x(k|t),u(k|t)) \\
  &x(t|t) = x(t) \\
  &x(k|t)\in X \\
  &u(k|t)\in U \\
  &x(t+N|t) = x_s
 \end{split}
\end{equation*}

Remark: this can be extended to terminal region/cost framework.

Closed-loop average perfomance

\begin{Theorem}
 The closed-loop asymptatic average perfomance is at least as good as the
 optimal steady-state cost i.e.,

 $$\lim_{T\rightarrow\infty}\sup\frac{\sum_{k=0}^{T-1}(x(k),u(k)}{T}\le L(x_s,u_s)$$

 \begin{proof}
  Lets present Imagine a completely new way of proving that we have not used before 
  in this course.

  Joke. We again consider difference $J^*(x(t+1))-J^*(x(t))$.

  $$J^*(x(t+1))-J^*(x(t))\le -L(x(t),u(t)) + L(x_s,u_s)$$

  From this equation we can get:

  $$\frac{\sum_{k=0}^{T-1}\left(J^*(x(k+1))-J^*(x(t))\right)}{T}\le\frac{\sum_{k=0}^{T-1}\left(L(x_s,u_s)+-L(x(t),u(t))\right)}{T}$$

  \begin{equation}\label{eq:EMPC1}
   \begin{split}
    \lim_{T\rightarrow\infty}\inf LHS \le &\lim_{T\rightarrow\infty}\inf RHS = \\
    &= L(x_s,u_s)+\lim{T\rightarrow\infty}\inf\frac{\sum_{k=0}^{T-1}-L(x(k),u(k))}{T} = \\
    &= L(x_s,u_s)-\lim{T\rightarrow\infty}\sup\frac{\sum_{k=0}^{T-1}L(x(k),u(k))}{T} 
   \end{split}
  \end{equation}

  Also, we can rewrite

  \begin{equation}\label{eq:EMPC2}
   \begin{split}
    \frac{\sum_{k=0}^{T-1}\left(J^*(x(k+1))-J^*(x(t))\right)}{T}=&\lim{T\rightarrow\infty}\inf\frac{J^*(x(T))-J^*(x(0))}{T} \ge \\
                       &\ge\lim{T\rightarrow\infty}\inf\frac{-J^*(x(0))}{T} \ge 0
   \end{split}
  \end{equation}

  Now combine (\ref{eq:EMPC1}) and (\ref{eq:EMPC2}) and we get

  $$\lim{T\rightarrow\infty}\sup\frac{\sum_{k=0}^{T-1}L(x(k),u(k))}{T} \le L(x_s,u_s)$$
 \end{proof}
\end{Theorem}


Classify optimal operating condition

\begin{Definition}
 A system is optimaly operated at steady-state if
 $$\lim{T\rightarrow\infty}\inf\frac{\sum_{k=0}^{T-1}L(x(k),u(k))}{T} \ge L(x_s,u_s)$$

 for all faesible sequences $(x(\cdot), u(\cdot))$.
\end{Definition}

\begin{Definition}
 A system is istrictly dissipative w.r.t. supply rate $s(x,u)$ if there exists a storage 
 function $\lambda:X\rightarrow\mathbb{R}_{\ge 0}$ s.t.
 $$\lambda(f(x,u))-\lambda(x)\le s(x,u)$$

 For strictlly dissipativity RHS=$s(x,u)-\rho((x-x_0))$, where $\rho$ positive defined.
\end{Definition}

\begin{Theorem}
 A system is optimally operated at steady state if it is dissipative w.r.t.
 $s(x,u)=L(x,u)-L(x_s,u_s)$.
 \begin{proof}
  \begin{equation}
   \begin{split}
    \lim_{T\rightarrow\infty}\inf\frac{\sum_{t=0}^{T-1}\lambda(x(t+1))-\lambda(x(t))}{T} &\le
           \lim_{T\rightarrow\infty}\inf\frac{\sum_{t=0}^{T-1}s(x(t),u(t))}{T}= \\
           &=\lim_{T\rightarrow\infty}\inf\frac{\sum_{t=0}^{T-1}L(x(t),u(t))}{T}-L(x_s,u_s)
   \end{split}
  \end{equation}

  Note, that $\lim_{T\rightarrow\infty}\inf\frac{\sum_{t=0}^{T-1}\lambda(x(t+1))-\lambda(x(t))}{T} \ge 0$
 \end{proof}
\end{Theorem}


\begin{Example}
 Consider system
 \begin{equation}
  \begin{split}
   &x(t+1)=x(t)u(t) \\
   &X=U=[-5,5] \\
   &L(x,u)=(x-1)^2+\delta(u-2)^2,\ \ 0<\delta<1
  \end{split}
 \end{equation}

 Set of all feasible steady-states ${(x,u)|x=0\ \ or \ \ u=1}$. Let $(x_s,u_s)=(1,1)$, then
 $L(x_s,u_s)=\delta$.

 $s(x,u)=L(x,u)-L(x_s,u_s)=(x-1)^2+\delta(u-2)^2-\delta$, we will search $\lambda(x)$ in form
 $$\lambda(x)=ax+c$$

 From dissipativity enequation we shooud have $\lambda(f(x,u))-\lambda(x)\le s(x,u)$, then
 compare $\lambda(f(x,u))-\lambda(x)=axu-ax$ with $s(x,u)=(x-1)^2+\delta(u-2)^2-\delta$.

 Consider $s(x,u)-(\lambda(f(x,u))-\lambda(x))=g(x,u)$ as function, and find parameters $a$ such
 that $g(x,u)\ge0$, $\forall x\in X,\ \ u\in U$. And minimum in point $(x_s,u_s)=(1,1)$
 $$\triangledown g(x,u)=\left[
  \begin{tabular}{l}
   $2(x+1)-au+a$ \\ 
   $2\delta(u-2) -a x$ 
  \end{tabular} 
 \right]$$

 Function $g$ is convex.

 $$\left[
  \begin{tabular}{l}
   $2(x+1)-au+a$ \\ 
   $2\delta(u-2) -a x$ 
  \end{tabular} 
 \right]_{(1,1)} = 0$$ 

 Thus $a = -2\delta$, and from requrements, that $\lambda(x)=ax+c\ge 0\ \ \forall x\in X=[-5,5]$
 we have $c\ge -ax = -2\delta x \ge [x=5] \ge -10\delta$.\

 And now we can use previous theorem to prove, that system is optimaly operated at
 steady state $(x_s,u_s)=(1,1)$.

 Consider case, when system is not dissipative. Change previous constraint for $u$, let $u\le 0$:
 then new optimal steady-state $(x_s,u_s)=(0,0)$ and $L(x_s,u_s)=4\delta+1$.

 Consider 2-period orbits:

 \begin{equation}
  \begin{split}
   &x(\cdot) = (1,-\frac{1}{3},1,-\frac{1}{3},\dots)\\
   &u(\cdot) = (-\frac{1}{3},-3,-\frac{1}{3},-3,\dots)
  \end{split}
 \end{equation}

 With this orbits

 $$\lim_{T\rightarrow\infty}\inf\frac{\sum_{t=0}^{T-1}L(x(t),u(t))}{T}=\frac{L(1,-\frac{1}{3})+L(-\frac{1}{3},-3)}{2}=
  = \frac{\frac{4}{3}^2}{2}+\delta\frac{\frac{-7}{3}+(-3)^2}{2}$$

  And for small enough $\delta$ this less then $L(0,0)$. That mean, that system is not optimal 
  operated at steady state and not dissipative w.r.t. chosen $s(x,u)$. 


\end{Example}
